\documentclass {article}
\usepackage {charter}

\begin {document}

\title {Introduction to Sensor Descriptive Movements}
\author {Zheng Lu, Zhiyang Zhang, Michael Craze}
\maketitle

\section {Background \& Motivation}

Basically, we are trying to find new ways to take advantage of sensors to improve our daily life quality. Since our movement information is crucial to implement some interesting applications, and most current work only focus on using GPS to get the movement or location information, which is neither energy efficient nor responsive. We want to develop an app that can translate our movement into readable format through sensor data.

\section {Our Idea}

Our idea is pretty intuitive,  we want to using sensor data to acquire readable movement information through some calculation on available sensor data on smartphones.
Potential interesting applications are listed here:

\begin {itemize}
	\item \textbf {Sensor-aided Path Recording} \\
		Current path recording apps such as "my tracks" are all using GPS transceivers to get users' location. Such a method has proven to be not very energy efficient. Smartphones are likely to running out of power in 1 or 2 hours. By using sensors we can greatly reduce the energy cost of these apps. The bigest challenge here is how to use low precision sensors available on smartphones to reduce the usage of GPS as much as possible.
	\item \textbf {Personal Aerial Photography} \\
		Sensors can provide way more responsive data than GPS system. Given the advantage of quick response of sensors, we can implement some real-time automatical control applications such as automatically control helicopters to give ourselves a live aerial videotyping.
\end {itemize}

\end {document}
